%╔════════════════════════════╗
%║	  Szablon dostosował	  ║
%║	mgr inż. Dawid Kotlarski  ║
%║		  06.10.2024		  ║
%╚════════════════════════════╝
\documentclass[12pt,twoside,a4paper,openany]{article}

    \input{preambula_pakiety.tex}
    \input{preambula_ustawienia.tex}

    %polecenia zdefiniowane w pakiecie strona_tytulowa.sty
    \title{Zaprojektować i wdrożyć system informatyczny na
    potrzeby przedsiębiorstwa zgodnie z założeniami}		%...Wpisać nazwę projektu...
    \author{Imie Nazwisko}
    \authorI{}
    \authorII{}		%jeśli są dwie osoby w projekcie to zostawiamy:    \authorII{}
		
	\uczelnia{AKADEMIA NAUK STOSOWANYCH \\W NOWYM SĄCZU}
    \instytut{Wydział Nauk Inżynieryjnych}
    \kierunek{Katedra Informatyki}
    \praca{DOKUMENTACJA PROJEKTOWA}
    \przedmiot{SYSTEMY OPERACYJNE}
    \prowadzacy{mgr inż. Jan Kozieński}
    \rok{2025}


%definicja składni mikrotik
\usepackage{fancyvrb}
\DefineVerbatimEnvironment{MT}{Verbatim}%
{commandchars=\+\[\],fontsize=\small,formatcom=\color{red},frame=lines,baselinestretch=1,} 
\let\mt\verb 
%zakonczenie definicji składni mikrotik

\usepackage{fancyhdr}    %biblioteka do nagłówka i stopki

			
\begin{document}
   
    \renewcommand{\figurename}{Rys.}    %musi byc pod \begin{document}, bo w~tym miejscu pakiet 'babel' narzuca swoje ustawienia
    \renewcommand{\tablename}{Tab.}     %j.w.
    \thispagestyle{empty}               %na tej stronie: brak numeru
    \stronatytulowa                     %strona tytułowa tworzona przez pakiet strona_tytulowa.tex
 
 \pagestyle{fancy}

    \newpage

    %formatowanie spisu treści i~nagłówków
    \renewcommand{\cftbeforesecskip}{8pt}
    \renewcommand{\cftsecafterpnum}{\vskip 8pt}
    \renewcommand{\cftparskip}{3pt}
    \renewcommand{\cfttoctitlefont}{\Large\bfseries\sffamily}
    \renewcommand{\cftsecfont}{\bfseries\sffamily}
    \renewcommand{\cftsubsecfont}{\sffamily}
    \renewcommand{\cftsubsubsecfont}{\sffamily}
    \renewcommand{\cftparafont}{\sffamily}
    %koniec formatowania spisu treści i nagłówków
     
    \tableofcontents    %spis treści
    \thispagestyle{fancy}
    \newpage

    
    \newpage

    
%%%%%%%%%%%%%%%%%%% treść główna dokumentu %%%%%%%%%%%%%%%%%%%%%%%%%

   %! SKRÓTY KLAWISZOWE
% LINK do skrótów klawiszowych: https://github.com/James-Yu/latex-workshop/wiki/Snippets
% ctrl+alt+j - przeniesienie z kodu do pdf
% ctrl + click - przeniesienie z pdf do kodu (dokument.pdf)
% zaznaczony fragment kodu -> ctrl+l -> ctrl+w
% gdy kuror na sekcji itp. -> cltr + alt + ] - obniżenie sekcji
% gdy kuror na sekcji itp. -> cltr + alt + [ - podniesienie sekcji
% kopia lini kodu -> ctrl + shift + strzałka w dół
\newpage
\section{Założenia projektowe – wymagania}		%1

% * SPOSOBY UZYWANIA MAKRA HERE * # 
% ? Listingi
% Parametr #1: Opis listingu (wyświetla sie bezpośrednio pod listingiem)
% Parametr #2 : Nazwa pliku oraz ID do oznaczania (wazne, zeby byl w katalogu kod oraz jego rozszerzenie to txt) 

\ListingFile{OpisPromptu}{prompt}

Tak wstawiamy listingi, za pomocą:

  \begin{verbatim}
	ListingFile\{Opis listingu}{nazwa-pliku} 
 \end{verbatim}
	
	wstawiamy listingi z plików z katalogu kod.

Gdzie:
\begin{verbatim}
	{Opis listingu} - opis listingu wyświetlany pod listingiem
	{nazwa-pliku} - nazwa pliku z katalogu kod i jednocześnie ID do oznaczania
\end{verbatim}

Tak oznaczamy listingi \OznaczKod{prompt} w tekście.\\Za pomocą:

\begin{verbatim}
	ListingFile\{Opis listingu}{nazwa-pliku} 
\end{verbatim}

gdzie:

\begin{verbatim}
	{nazwa-pliku} - nazwa pliku z katalogu kod i jednocześnie ID do oznaczania
\end{verbatim}



\clearpage
% ? Zdjęcia
% OPCJONALNY Parametr #1: Szerokość zdjęcia (domyślnie jest to szerokość paragrafu ale jak sie poda w [#1] to wtedy zmienia się na podaną wartość)
% Parametr #2: Nazwa pliku z rozszerzeniem (podajemy katalog w którym jest plik i jego rozszerzenie np. rys/nazwa-rysunku.png)
% Parametr #3: Opis tego co jest na rysunku (wyświetla sie bezpośrednio pod rysunkiem) 
% Parametr #4: Identyfikator rysunku (do oznaczania zdjęć w tekście) 

\fg{rys/nazwa-rysunku.png}{Opis tego co jest na rysunku}{id-rysunku}

jest równe 
\fg[\textwidth]{rys/nazwa-rysunku.png}{Opis tego co jest na rysunku}{id-rysunku}
\clearpage
Tak wstawiamy zdjęcia, za pomocą:

  \begin{verbatim}
	\fg{szerokość-rysunku}{nazwa-pliku}{Opis rysunku}{id-rysunku} 
 \end{verbatim}

 Gdzie:
\begin{verbatim}
	{szerokość-rysunku} - szerokość rysunku domyślnie \textwidth
	{nazwa-pliku} - nazwa pliku z katalogu rys i jego rozszerzenie
	{Opis rysunku} - opis rysunku wyświetlany pod rysunkiem
	{id-rysunku} - identyfikator rysunku
\end{verbatim}

 Tak oznaczamy zdjęcia \OznaczZdjecie{id-rysunku} w tekście. Za pomocą:

  \begin{verbatim}
	\OznaczZdjecie{id-rysunku}
	\end{verbatim}




   \newpage
\section{Opis użytych technologii}		%2
%(W podpunktach dokonać krótkiej charakterystyki użytych technologii )

\subsection{Active Directory}
Active Directory (AD) to usługa katalogowa opracowana przez firmę Microsoft dla systemów Windows. Umożliwia zarządzanie użytkownikami, komputerami i innymi zasobami w sieci. AD zapewnia centralne uwierzytelnianie i autoryzację, co ułatwia zarządzanie dużymi sieciami.

\subsection{IIS (Internet Information Services)}
Internet Information Services (IIS) to serwer WWW stworzony przez Microsoft. Umożliwia hostowanie stron internetowych oraz aplikacji webowych. IIS obsługuje różne technologie, takie jak ASP.NET, PHP i WordPress, co czyni go wszechstronnym narzędziem do wdrażania aplikacji internetowych.

\subsection{WordPress}
WordPress to popularny system zarządzania treścią (CMS), który umożliwia tworzenie i zarządzanie stronami internetowymi. Jest łatwy w użyciu i oferuje szeroką gamę wtyczek oraz motywów, co pozwala na dostosowanie strony do indywidualnych potrzeb.

\subsection{DHCP (Dynamic Host Configuration Protocol)}
Dynamic Host Configuration Protocol (DHCP) to protokół sieciowy, który automatycznie przydziela adresy IP oraz inne parametry konfiguracyjne urządzeniom w sieci. DHCP upraszcza zarządzanie adresacją IP, eliminując konieczność ręcznego przypisywania adresów.

\subsection{RAID (Redundant Array of Independent Disks)}
RAID to technologia, która łączy wiele dysków twardych w jedną logiczną jednostkę w celu zwiększenia wydajności i/lub zapewnienia redundancji danych. RAID-1, znany również jako mirroring, polega na duplikowaniu danych na dwóch dyskach, co zapewnia ochronę przed utratą danych w przypadku awarii jednego z dysków.

\subsection{iSCSI (Internet Small Computer Systems Interface)}
iSCSI to protokół sieciowy, który umożliwia przesyłanie poleceń SCSI przez sieci IP. Dzięki iSCSI można tworzyć rozproszone systemy pamięci masowej, które są dostępne przez sieć, co pozwala na centralizację zarządzania danymi i zwiększenie elastyczności infrastruktury IT.

\subsection{GPO (Group Policy Objects)}
Group Policy Objects (GPO) to funkcja systemów Windows, która umożliwia centralne zarządzanie konfiguracją i ustawieniami komputerów oraz użytkowników w domenie Active Directory. GPO pozwala na automatyzację wielu zadań administracyjnych, takich jak instalacja oprogramowania, konfiguracja systemu i ustawienia zabezpieczeń.

\subsection{DNS (Domain Name System)}
Domain Name System (DNS) to system, który przyporządkowuje adresy IP do nazw domenowych. DNS ułatwia korzystanie z zasobów sieciowych, ponieważ pozwala na odwoływanie się do serwerów i usług za pomocą nazw zamiast adresów IP. Dzięki DNS możliwe jest również tworzenie hierarchicznej struktury domen, co ułatwia zarządzanie dużymi sieciami.

\subsection{FS (File Server)}
File Server (FS) to serwer, który umożliwia przechowywanie i udostępnianie plików w sieci. FS zapewnia centralizację danych, co ułatwia zarządzanie zasobami oraz zapewnia bezpieczeństwo i dostępność danych. FS może być wykorzystywany do przechowywania plików użytkowników, kopii zapasowych oraz innych danych firmowych.

\subsection{WDS (Windows Deployment Services)}
Windows Deployment Services (WDS) to narzędzie, które umożliwia zdalną instalację systemu Windows na stacjach roboczych. WDS pozwala na automatyzację procesu instalacji, co przyspiesza wdrażanie nowych komputerów oraz ułatwia zarządzanie systemami w sieci. WDS współpracuje z technologią PXE, która umożliwia uruchamianie systemu z sieci.

\subsection{iSCSI Target}
iSCSI Target to serwer, który udostępnia pamięć masową za pomocą protokołu iSCSI. iSCSI Target pozwala na tworzenie wirtualnych dysków, które są dostępne przez sieć i mogą być wykorzystywane przez inne serwery i stacje robocze. Dzięki iSCSI Target możliwe jest tworzenie rozproszonych systemów pamięci masowej, które są łatwo skalowalne i elastyczne. 

   	\newpage
\section{Schemat logiczny}		%3
% (Ma zawierać aktualne nazewnictwo i adresację IP.)
   \newpage
\section{Procedury instalacyjne poszczególnych usług}		%4
% Procedury instalacyjne poszczególnych usług.
% (W podpunktach zamieścić polecenia dotyczące instalacji wdrażanych usług) 
	\subsection{Instalacja serwera sdc}
		\fg{rys/1. Instalacja Serwera/1.png}{Tworzenie maszyny wirtualnej serwera sdc}{1} 
		\clearpage
		\fg{rys/1. Instalacja Serwera/2.png}{}{2} 
		\clearpage
		\fg{rys/1. Instalacja Serwera/3.png}{}{3}
		\clearpage
		\fg{rys/1. Instalacja Serwera/4.png}{Konfiguracja parametrów maszyny}{4}
		\clearpage
		\fg{rys/1. Instalacja Serwera/5.png}{Konfiguracja parametrów maszyny}{5}
		\clearpage
		\fg{rys/1. Instalacja Serwera/6.png}{}{6}
		\clearpage
		\fg{rys/1. Instalacja Serwera/7.png}{}{7}
		\clearpage
		\fg{rys/1. Instalacja Serwera/8.png}{Instalacja windows serwer}{8}
		\clearpage
		\fg{rys/1. Instalacja Serwera/9.png}{}{9}
		\clearpage
		\fg{rys/1. Instalacja Serwera/10.png}{Wybór prawidłowej wersji windowsa}{10}
		\clearpage
		\fg{rys/1. Instalacja Serwera/11.png}{}{11}
		\clearpage
		\fg{rys/1. Instalacja Serwera/12.png}{}{12}
		\clearpage
		\fg{rys/1. Instalacja Serwera/13.png}{}{13}
		\clearpage
		\fg{rys/1. Instalacja Serwera/14.png}{}{14}
		\clearpage
		\fg{rys/1. Instalacja Serwera/15.png}{}{15}
		\clearpage
		\fg{rys/1. Instalacja Serwera/16.png}{Widok zainstalowanego systemu}{16}
		\clearpage
		\fg{rys/1. Instalacja Serwera/17.png}{Konfiguracja karty}{17}
		\clearpage
		\fg{rys/1. Instalacja Serwera/18.png}{Instalacja Polskiej wercji językowej}{18}
		\clearpage
		\fg{rys/1. Instalacja Serwera/19.png}{}{19}
		\clearpage
		\fg{rys/1. Instalacja Serwera/22.png}{Zmiana nazwy maszyny}{22}
		\clearpage
		\fg{rys/1. Instalacja Serwera/26.png}{Konfiguracja karty sieciowej}{26}
		\clearpage
	
	\subsection{Instalacja Active Directory}
		\fg{rys/2. Active Directory/1.png}{Dodawanie roli}{1}
		\clearpage
		\fg{rys/2. Active Directory/2.png}{}{2}
		\clearpage
		\fg{rys/2. Active Directory/3.png}{}{3}
		\clearpage
		\fg{rys/2. Active Directory/4.png}{}{4}
		\clearpage
		\fg{rys/2. Active Directory/5.png}{Wybór AD}{5}
		\clearpage
		\fg{rys/2. Active Directory/6.png}{Wybór roli serwera dns}{6}
		\clearpage
		\fg{rys/2. Active Directory/7.png}{}{7}
		\clearpage
		\fg{rys/2. Active Directory/8.png}{Instalacja ról}{8}
		\clearpage
		\fg{rys/2. Active Directory/9.png}{}{9}
		\clearpage
		\fg{rys/2. Active Directory/10.png}{Początkowa konfiguracja serwera AD}{10}
		\clearpage
		\fg{rys/2. Active Directory/11.png}{}{11}
		\clearpage
		\fg{rys/2. Active Directory/12.png}{Ustawienie nazwy domeny na doit}{12}
		\clearpage
		\fg{rys/2. Active Directory/13.png}{}{13}
		\clearpage
		\fg{rys/2. Active Directory/14.png}{}{14}
		\clearpage
		\fg{rys/2. Active Directory/15.png}{Instalacja usług domenowych}{15}
		\clearpage
		\fg{rys/2. Active Directory/16.png}{}{16}
		\clearpage
		\fg{rys/2. Active Directory/17.png}{Widok użytkowników dodawanych przez skrypt}{17}
		\clearpage
		\fg{rys/2. Active Directory/18.png}{Wynik działania skryptu}{18}
		\clearpage
		\fg{rys/2. Active Directory/19.png}{Utworzone przez skrypt grupy}{19}
		\clearpage
		\fg{rys/2. Active Directory/20.png}{Oraz użytkownicy tych grup}{20}
		\clearpage
		\fg{rys/2. Active Directory/21.png}{Prezentacja członków jednej z grup}{21}
		\clearpage
		\fg{rys/2. Active Directory/22.png}{}{22}
		\clearpage
		\fg{rys/2. Active Directory/23.png}{}{23}
		\clearpage

	\subsection{Instalacja RAID 1}
	\fg{rys/3. raid1/1.png}{Dodawanie dysku do maszyny}{1}
	\clearpage
	\fg{rys/3. raid1/2.png}{Dodawanie dysku już w maszynie}{2}
	\clearpage
	\fg{rys/3. raid1/3.png}{Tworzenie raid1}{3}
	\clearpage
	\fg{rys/3. raid1/4.png}{Wybór dysków}{4}
	\clearpage
	\fg{rys/3. raid1/5.png}{}{5}
	\clearpage
	\fg{rys/3. raid1/6.png}{}{6}
	\clearpage
	\fg{rys/3. raid1/7.png}{Akceptacja konfiguracji}{7}
	\clearpage
	\fg{rys/3. raid1/8.png}{}{8}
	\clearpage
	\fg{rys/3. raid1/9.png}{}{9}
	\clearpage
	\fg{rys/3. raid1/10.png}{Zmiana litery dysku}{10}
	\clearpage
	\fg{rys/3. raid1/11.png}{}{11}
	\clearpage
	\fg{rys/3. raid1/12.png}{}{12}
	\clearpage
	\fg{rys/3. raid1/13.png}{Finalizacja konfiguracji RAID 1}{13}
	\clearpage
	\newpage

	\subsection{Instalacja iSCSI}
	\fg{rys/4. iSCSI/1.png}{Dodawanie roli iSCSI Target serwer}{1}
	\clearpage
	\fg{rys/4. iSCSI/2.png}{}{2}
	\clearpage
	\fg{rys/4. iSCSI/3.png}{}{3}
	\clearpage
	\fg{rys/4. iSCSI/4.png}{Rozpoczęcie instalacji}{4}
	\clearpage
	\fg{rys/4. iSCSI/5.png}{}{5}
	\clearpage
	\fg{rys/4. iSCSI/6.png}{Wybór dysku}{6}
	\clearpage
	\fg{rys/4. iSCSI/7.png}{Ustawienie własnej nazwy klastra}{7}
	\clearpage
	\fg{rys/4. iSCSI/8.png}{Wybór rozmiaru klastra}{8}
	\clearpage
	\fg{rys/4. iSCSI/9.png}{}{9}
	\clearpage
	\fg{rys/4. iSCSI/10.png}{}{10}
	\clearpage
	\fg{rys/4. iSCSI/11.png}{}{11}
	\clearpage
	\fg{rys/4. iSCSI/12.png}{}{12}
	\clearpage
	\fg{rys/4. iSCSI/13.png}{}{13}
	\clearpage
	\fg{rys/4. iSCSI/14.png}{}{14}
	\clearpage
	\fg{rys/4. iSCSI/15.png}{Akceptacja konfiguracji}{15}
	\clearpage
	\fg{rys/4. iSCSI/16.png}{}{16}
	\clearpage
	\fg{rys/4. iSCSI/17.png}{Widok zainstalowanego klastra}{17}
	\clearpage
	\fg{rys/4. iSCSI/18.png}{Konfiguracja serwerów pomocniczych}{18}
	\clearpage
	\fg{rys/4. iSCSI/19.png}{}{19}
	\clearpage
	\fg{rys/4. iSCSI/20.png}{Konfiguracja kart sieciowych}{20}
	\clearpage
	\fg{rys/4. iSCSI/21.png}{Dołączanie do domeny}{21}
	\clearpage
	\fg{rys/4. iSCSI/22.png}{}{22}
	\clearpage
	\fg{rys/4. iSCSI/23.png}{}{23}
	\clearpage
	\fg{rys/4. iSCSI/24.png}{}{24}
	\clearpage
	\fg{rys/4. iSCSI/25.png}{}{25}
	\clearpage
	\fg{rys/4. iSCSI/26.png}{}{26}
	\clearpage
	\fg{rys/4. iSCSI/27.png}{}{27}
	\clearpage
	\fg{rys/4. iSCSI/28.png}{}{28}
	\clearpage
	\fg{rys/4. iSCSI/29.png}{}{29}
	\clearpage
	\fg{rys/4. iSCSI/30.png}{}{30}
	\clearpage
	\fg{rys/4. iSCSI/31.png}{}{31}
	\clearpage
	\fg{rys/4. iSCSI/32.png}{}{32}
	\clearpage
	\fg{rys/4. iSCSI/33.png}{}{33}
	\clearpage
	\fg{rys/4. iSCSI/34.png}{Dodawanie roli na serwerach pomocniczych}{34}
	\clearpage
	\fg{rys/4. iSCSI/35.png}{}{35}
	\clearpage
	\fg{rys/4. iSCSI/36.png}{Wszystko dzieje się już z serwera głównego}{36}
	\clearpage
	\fg{rys/4. iSCSI/37.png}{}{37}
	\clearpage
	\fg{rys/4. iSCSI/38.png}{}{38}
	\clearpage
	\fg{rys/4. iSCSI/39.png}{Podłączanie dysku}{39}
	\clearpage
	\fg{rys/4. iSCSI/40.png}{Konfiguracja woluminu}{40}
	\clearpage
	\fg{rys/4. iSCSI/41.png}{}{41}
	\clearpage
	\fg{rys/4. iSCSI/42.png}{Finalizacja konfiguracji}{42}
	\clearpage
	\fg{rys/4. iSCSI/43.png}{Efekty}{43}
	\clearpage
	\fg{rys/4. iSCSI/44.png}{Konfiguracja klastra na sn1}{44}
	\clearpage
	\fg{rys/4. iSCSI/45.png}{}{45}
	\clearpage
	\fg{rys/4. iSCSI/46.png}{}{46}
	\clearpage
	\fg{rys/4. iSCSI/47.png}{}{47}
	\clearpage
	\fg{rys/4. iSCSI/48.png}{}{48}
	\clearpage
	\fg{rys/4. iSCSI/49.png}{Wybór adresu IP}{49}
	\clearpage
	\fg{rys/4. iSCSI/50.png}{Potwierdzanie ustawień}{50}
	\clearpage
	\fg{rys/4. iSCSI/51.png}{}{51}
	\clearpage

	\newpage
	\subsection{instalacja FS}
	\fg{rys/5. FS/1.png}{Instalacja FS z serwera SDC}{1}
	\clearpage
	\fg{rys/5. FS/2.png}{}{2}
	\clearpage
	\fg{rys/5. FS/3.png}{}{3}
	\clearpage
	\fg{rys/5. FS/4.png}{}{4}
	\clearpage
	\fg{rys/5. FS/5.png}{Dodawanie nowego dysku wirtualnego dls FS}{5}
	\clearpage
	\fg{rys/5. FS/6.png}{Wybór dysku}{6}
	\clearpage
	\fg{rys/5. FS/7.png}{Wybór nazwy}{7}
	\clearpage
	\fg{rys/5. FS/8.png}{Ustalenie rozmiaru}{8}
	\clearpage
	\fg{rys/5. FS/9.png}{Tworzenie nowego obiektu docelowego}{9}
	\clearpage
	\fg{rys/5. FS/10.png}{Określenie nazwy}{10}
	\clearpage
	\fg{rys/5. FS/11.png}{Wybór serweru dostępu}{11}
	\clearpage
	\fg{rys/5. FS/12.png}{}{12}
	\clearpage
	\fg{rys/5. FS/13.png}{Wyniki instalacji}{13}
	\clearpage
	\fg{rys/5. FS/14.png}{}{14}
	\clearpage
	\fg{rys/5. FS/15.png}{Dodajemy file server}{15}
	\clearpage
	\fg{rys/5. FS/16.png}{Konfiguracja FS}{16}
	\clearpage
	\fg{rys/5. FS/17.png}{Wybór adresu ip serwera plików}{17}
	\clearpage
	\fg{rys/5. FS/18.png}{Dodanie FS przez iSCSI aby był widoczny}{18}
	\clearpage
	\fg{rys/5. FS/19.png}{}{19}
	\clearpage
	\fg{rys/5. FS/20.png}{}{20}
	\clearpage
	\fg{rys/5. FS/21.png}{Inicjalizacja nowego dysku pod FS}{21}
	\clearpage
	\fg{rys/5. FS/22.png}{}{22}
	\clearpage
	\fg{rys/5. FS/23.png}{}{23}
	\clearpage
	\fg{rys/5. FS/24.png}{}{24}
	\clearpage
	\fg{rys/5. FS/25.png}{}{25}
	\clearpage
	\fg{rys/5. FS/26.png}{}{26}
	\clearpage
	\fg{rys/5. FS/27.png}{}{27}
	\clearpage
	\fg{rys/5. FS/28.png}{}{28}
	\clearpage
	\fg{rys/5. FS/29.png}{Zatwierdzenie ustawień}{29}
	\clearpage
	\fg{rys/5. FS/30.png}{}{30}
	\clearpage
	\fg{rys/5. FS/31.png}{Teraz trzeba będzie naprawić widoczny błąd}{31}
	\clearpage
	\fg{rys/5. FS/32.png}{Tu jest rozwiązanie problemu który się pojawił}{32}
	\clearpage
	\fg{rys/5. FS/33.png}{Tworzenie folderu wspólnego}{33}
	\clearpage
	\fg{rys/5. FS/34.png}{}{34}
	\clearpage
	\fg{rys/5. FS/35.png}{}{35}
	\clearpage
	\fg{rys/5. FS/36.png}{}{36}
	\clearpage
	\fg{rys/5. FS/37.png}{Edycja uprawnień}{37}
	\clearpage
	\fg{rys/5. FS/38.png}{Akceptacja ustawień}{38}
	\clearpage
	\fg{rys/5. FS/39.png}{}{39}
	\clearpage
	\fg{rys/5. FS/40.png}{Wynik tworzenia dysku wspólnego}{40}
	\clearpage
	\fg{rys/5. FS/41.png}{Wynik na kliencie}{41}
	\clearpage
	\fg{rys/5. FS/42.png}{Działania powtarzane dla pozostałych folderów}{42}
	\clearpage
	\fg{rys/5. FS/43.png}{}{43}
	\clearpage
	\fg{rys/5. FS/44.png}{Edycja uprawnień}{44}
	\clearpage
	\fg{rys/5. FS/45.png}{}{45}
	\clearpage
	\fg{rys/5. FS/46.png}{}{46}
	\clearpage
	\fg{rys/5. FS/47.png}{Wyniki}{47}
	\clearpage

	\newpage
	\subsection{Instalacja GPO}
	\fg{rys/6. GPO/1.png}{Tworzenie obiektu zasad}{1}
	\clearpage
	\fg{rys/6. GPO/2.png}{Tworzenie obiektu zasad grupy}{2}
	\clearpage
	\fg{rys/6. GPO/3.png}{Edycja obiektu}{3}
	\clearpage
	\fg{rys/6. GPO/4.png}{Dodawanie mapowanego dysku dla poszczególnych folderów}{4}
	\clearpage
	\fg{rys/6. GPO/5.png}{}{5}
	\clearpage
	\fg{rys/6. GPO/6.png}{}{6}
	\clearpage
	\fg{rys/6. GPO/7.png}{Aktualizacja gpo w celu zaobserwowania dodanych przez nas folderów}{7}
	\clearpage
	\fg{rys/6. GPO/8.png}{zasada Dysk dla programistów}{8}
	\clearpage
	\fg{rys/6. GPO/9.png}{}{9}
	\clearpage
	\fg{rys/6. GPO/10.png}{}{10}
	\clearpage
	\fg{rys/6. GPO/11.png}{}{11}
	\clearpage
	\fg{rys/6. GPO/12.png}{}{12}
	\clearpage
	\fg{rys/6. GPO/13.png}{Wynik}{13}
	\clearpage
	\fg{rys/6. GPO/14.png}{}{14}
	\clearpage
	\fg{rys/6. GPO/15.png}{}{15}
	\clearpage
	\fg{rys/6. GPO/16.png}{Konfiguracja dla księgowych}{16}
	\clearpage
	\fg{rys/6. GPO/17.png}{}{17}
	\clearpage
	\fg{rys/6. GPO/18.png}{Zmiana właściwości dysku}{18}
	\clearpage
	\fg{rys/6. GPO/19.png}{konfiguracja zabezpieczneń}{19}
	\clearpage
	\fg{rys/6. GPO/20.png}{}{20}
	\clearpage
	\fg{rys/6. GPO/21.png}{Dodanie zmiennej środowiskowej}{21}
	\clearpage
	\fg{rys/6. GPO/22.png}{Konfiguracja jej właściwości}{22}
	\clearpage
	\fg{rys/6. GPO/23.png}{Wyniki zmiany dodawania tej zmiennej}{23}
	\clearpage
	\fg{rys/6. GPO/24.png}{Utworzenie nowego publicznego folderu}{24}
	\clearpage
	\fg{rys/6. GPO/25.png}{}{25}
	\clearpage
	\fg{rys/6. GPO/26.png}{}{26}
	\clearpage
	\fg{rys/6. GPO/27.png}{Aktualizacja gpo}{27}
	\clearpage
	\fg{rys/6. GPO/28.png}{Wynik na innej maszynie}{28}
	\clearpage

	\newpage
	\subsection{Instalacja DNS}
	\fg{rys/7. DNS/1.png}{Wejście do konfiguracji dnsa}{1}
	\clearpage
	\fg{rys/7. DNS/2.png}{Dodawanie nowego hosta}{2}
	\clearpage
	\fg{rys/7. DNS/3.png}{Konfiguracja hosta}{3}
	\clearpage
	\fg{rys/7. DNS/4.png}{Konfiguracja strefy przeszukiwania wstecznego}{4}
	\clearpage
	\fg{rys/7. DNS/5.png}{}{5}
	\clearpage
	\fg{rys/7. DNS/6.png}{}{6}
	\clearpage
	\fg{rys/7. DNS/7.png}{}{7}
	\clearpage
	\fg{rys/7. DNS/8.png}{}{8}
	\clearpage
	\fg{rys/7. DNS/9.png}{Konfiguracja adresu}{9}
	\clearpage
	\fg{rys/7. DNS/10.png}{}{10}
	\clearpage
	\fg{rys/7. DNS/11.png}{Zatwierdzenie konfiguracji}{11}
	\clearpage
	\fg{rys/7. DNS/12.png}{Dodawanie nowego wskaźnika PTR}{12}
	\clearpage
	\fg{rys/7. DNS/13.png}{Wybór serwera}{13}
	\clearpage
	\fg{rys/7. DNS/14.png}{}{14}
	\clearpage
	\fg{rys/7. DNS/15.png}{Określenie adresu ip}{15}
	\clearpage

	\newpage
	\subsection{Instalacja DHCP}
	\fg{rys/8. DHCP/1.png}{Rozpoczęcie konfiguracji w failover cluster}{1}
	\clearpage
	\fg{rys/8. DHCP/2.png}{}{2}
	\clearpage
	\fg{rys/8. DHCP/3.png}{Wybór serwera DHCP}{3}
	\clearpage
	\fg{rys/8. DHCP/4.png}{Rozpoczęcie instalacji na maszynach SN1 i SN2}{4}
	\clearpage
	\fg{rys/8. DHCP/5.png}{}{5}
	\clearpage
	\fg{rys/8. DHCP/6.png}{}{6}
	\clearpage
	\fg{rys/8. DHCP/7.png}{}{7}
	\clearpage
	\fg{rys/8. DHCP/8.png}{Rozpoczęcie konfiguracji na SN1}{8}
	\clearpage
	\fg{rys/8. DHCP/9.png}{Wybór odpowiedniego adresu IP}{9}
	\clearpage
	\fg{rys/8. DHCP/10.png}{Utworzenie dysku dla serwera DHCP}{10}
	\clearpage
	\fg{rys/8. DHCP/11.png}{Dodanie dysku do maszyny}{11}
	\clearpage
	\fg{rys/8. DHCP/12.png}{}{12}
	\clearpage
	\fg{rys/8. DHCP/13.png}{Formatowanie dysku}{13}
	\clearpage
	\fg{rys/8. DHCP/14.png}{Dodanie dysku jako dysk wirtualny}{14}
	\clearpage
	\fg{rys/8. DHCP/15.png}{}{15}
	\clearpage
	\fg{rys/8. DHCP/16.png}{Określenie nazwy dysku wirtualnego}{16}
	\clearpage
	\fg{rys/8. DHCP/17.png}{Wybór rozmiaru}{17}
	\clearpage
	\fg{rys/8. DHCP/18.png}{}{18}
	\clearpage
	\fg{rys/8. DHCP/19.png}{Określenie nazwy obiektu docelowego}{19}
	\clearpage
	\fg{rys/8. DHCP/20.png}{}{20}
	\clearpage
	\fg{rys/8. DHCP/21.png}{Sprawdzenie i Potwierdzanie ustawień}{21}
	\clearpage
	\fg{rys/8. DHCP/22.png}{Podłączanie dysku na maszynie SN1}{22}
	\clearpage
	\fg{rys/8. DHCP/23.png}{}{23}
	\clearpage
	\fg{rys/8. DHCP/24.png}{}{24}
	\clearpage
	\fg{rys/8. DHCP/25.png}{Dodanie dysku do systemu}{25}
	\clearpage
	\fg{rys/8. DHCP/26.png}{}{26}
	\clearpage
	\fg{rys/8. DHCP/27.png}{}{27}
	\clearpage
	\fg{rys/8. DHCP/28.png}{}{28}
	\clearpage
	\fg{rys/8. DHCP/29.png}{Konfiguracja dysku w klastrze}{29}
	\clearpage
	\fg{rys/8. DHCP/30.png}{}{30}
	\clearpage
	\fg{rys/8. DHCP/31.png}{}{31}
	\clearpage
	\fg{rys/8. DHCP/32.png}{}{32}
	\clearpage
	\fg{rys/8. DHCP/33.png}{Konfiguracja serwera DHCP na SN1}{33}
	\clearpage
	\fg{rys/8. DHCP/34.png}{Określenie zakresu}{34}
	\clearpage
	\fg{rys/8. DHCP/35.png}{}{35}
	\clearpage
	\fg{rys/8. DHCP/36.png}{Wybór odpowiedniej puli adresów}{36}
	\clearpage
	\fg{rys/8. DHCP/37.png}{}{37}
	\clearpage
	\fg{rys/8. DHCP/38.png}{}{38}
	\clearpage
	\fg{rys/8. DHCP/39.png}{}{39}
	\clearpage
	\fg{rys/8. DHCP/40.png}{}{40}
	\clearpage
	\fg{rys/8. DHCP/41.png}{}{41}
	\clearpage
	\fg{rys/8. DHCP/42.png}{}{42}
	\clearpage
	\fg{rys/8. DHCP/43.png}{}{43}
	\clearpage
	\fg{rys/8. DHCP/44.png}{Zmieniamy ustawienia adresu ip na DHCP na maszynie klienta}{44}
	\clearpage
	\fg{rys/8. DHCP/45.png}{Autoryzujemy serwer DHCP na SN1 oraz na SDC}{45}
	\clearpage

	\newpage
	\subsection{Instalacja WDS}
	\fg{rys/9. WDS/1.png}{Rozpoczęcie instalacji serwera WDS}{1}
	\clearpage
	\fg{rys/9. WDS/2.png}{}{2}
	\clearpage
	\fg{rys/9. WDS/3.png}{}{3}
	\clearpage
	\fg{rys/9. WDS/4.png}{}{4}
	\clearpage
	\fg{rys/9. WDS/5.png}{}{5}
	\clearpage
	\fg{rys/9. WDS/6.png}{}{6}
	\clearpage
	\fg{rys/9. WDS/7.png}{}{7}
	\clearpage
	\fg{rys/9. WDS/8.png}{Rozpoczęcie konfiguracji serwera WDS}{8}
	\clearpage
	\fg{rys/9. WDS/9.png}{}{9}
	\clearpage
	\fg{rys/9. WDS/10.png}{}{10}
	\clearpage
	\fg{rys/9. WDS/11.png}{}{11}
	\clearpage
	\fg{rys/9. WDS/12.png}{Wybór pliku obrazu}{12}
	\clearpage
	\fg{rys/9. WDS/13.png}{}{13}
	\clearpage
	\fg{rys/9. WDS/14.png}{Kopiowanie plików z dysku ISO z systemem}{14}
	\clearpage
	\fg{rys/9. WDS/15.png}{Wklejenie ich do określonej lokalizacji}{15}
	\clearpage
	\fg{rys/9. WDS/16.png}{Wybranie pliku obrazu}{16}
	\clearpage
	\fg{rys/9. WDS/17.png}{Wybór znajdującego się na nim systemu}{17}
	\clearpage
	\fg{rys/9. WDS/18.png}{}{18}
	\clearpage
	\fg{rys/9. WDS/19.png}{}{19}
	\clearpage
	\fg{rys/9. WDS/20.png}{Wybór obrazu rozruchowego}{20}
	\clearpage
	\fg{rys/9. WDS/21.png}{}{21}
	\clearpage
	\fg{rys/9. WDS/22.png}{}{22}
	\clearpage
	\fg{rys/9. WDS/23.png}{}{23}
	\clearpage
	\fg{rys/9. WDS/24.png}{}{24}
	\clearpage
	\fg{rys/9. WDS/25.png}{}{25}
	\clearpage
	\fg{rys/9. WDS/26.png}{}{26}
	\clearpage
	\fg{rys/9. WDS/27.png}{Utworzenie nowej maszyny}{27}
	\clearpage
	\fg{rys/9. WDS/28.png}{Konfiguracja ustawień}{28}
	\clearpage
	\fg{rys/9. WDS/29.png}{Konfiguracja ustawień sieciowych}{29}
	\clearpage

	\newpage
	\subsection{Instalacja serwera drukowania}
	\fg{rys/10. Drukarka/1.png}{Instalacja Print and Document service}{1}
	\clearpage
	\fg{rys/10. Drukarka/2.png}{}{2}
	\clearpage
	\fg{rys/10. Drukarka/3.png}{}{3}
	\clearpage
	\fg{rys/10. Drukarka/4.png}{Wybór serwera wydruku}{4}
	\clearpage
	\fg{rys/10. Drukarka/5.png}{}{5}
	\clearpage
	\fg{rys/10. Drukarka/6.png}{Konfiguracja serwera}{6}
	\clearpage
	\fg{rys/10. Drukarka/7.png}{}{7}
	\clearpage
	\fg{rys/10. Drukarka/8.png}{Dodawanie drukarki}{8}
	\clearpage
	\fg{rys/10. Drukarka/9.png}{Instalacja sterownika drukarki}{9}
	\clearpage
	\fg{rys/10. Drukarka/10.png}{}{10}
	\clearpage
	\fg{rys/10. Drukarka/12.png}{}{12}
	\clearpage
	\fg{rys/10. Drukarka/13.png}{}{13}
	\clearpage
	\fg{rys/10. Drukarka/14.png}{}{14}
	\clearpage
	\fg{rys/10. Drukarka/15.png}{Widok drukarki na serwerze głównym}{15}
	\clearpage
	\fg{rys/10. Drukarka/16.png}{Udostępnienie drukarki w sieci}{16}
	\clearpage
	\fg{rys/10. Drukarka/17.png}{}{17}
	\clearpage

	\newpage
	\subsection{Instalacja IIS}
	\fg{rys/11. IIS/1.png}{}{1}
	\clearpage
	\fg{rys/11. IIS/2.png}{}{2}
	\clearpage
	\fg{rys/11. IIS/3.png}{}{3}
	\clearpage
	\fg{rys/11. IIS/4.png}{}{4}
	\clearpage
	\fg{rys/11. IIS/5.png}{}{5}
	\clearpage
	\fg{rys/11. IIS/6.png}{}{6}
	\clearpage
	\fg{rys/11. IIS/7.png}{}{7}
	\clearpage
	\fg{rys/11. IIS/8.png}{}{8}
	\clearpage
	\fg{rys/11. IIS/9.png}{}{9}
	\clearpage
	\fg{rys/11. IIS/10.png}{}{10}
	\clearpage
	\fg{rys/11. IIS/11.png}{}{11}
	\clearpage
	\fg{rys/11. IIS/12.png}{}{12}
	\clearpage
	\fg{rys/11. IIS/13.png}{}{13}
	\clearpage
	\fg{rys/11. IIS/14.png}{}{14}
	\clearpage
	\fg{rys/11. IIS/15.png}{}{15}
	\clearpage
	\fg{rys/11. IIS/16.png}{}{16}
	\clearpage
	\fg{rys/11. IIS/17.png}{}{17}
	\clearpage
	\fg{rys/11. IIS/18.png}{}{18}
	\clearpage

	\newpage
	\subsection{Instalacja Wordpress}
	\fg{rys/12. Wordpress/1.png}{}{1}	
	\clearpage
	\fg{rys/12. Wordpress/2.png}{}{2}
	\clearpage
	\fg{rys/12. Wordpress/3.png}{}{3}
	\clearpage
	\fg{rys/12. Wordpress/4.png}{}{4}
	\clearpage
	\fg{rys/12. Wordpress/5.png}{}{5}
	\clearpage
	\fg{rys/12. Wordpress/6.png}{}{6}
	\clearpage
	\fg{rys/12. Wordpress/7.png}{}{7}
	\clearpage
	\fg{rys/12. Wordpress/8.png}{}{8}
	\clearpage
	\fg{rys/12. Wordpress/9.png}{}{9}
	\clearpage
	\fg{rys/12. Wordpress/10.png}{}{10}
	\clearpage
	\fg{rys/12. Wordpress/11.png}{}{11}
	\clearpage
	\fg{rys/12. Wordpress/12.png}{}{12}
	\clearpage
	\fg{rys/12. Wordpress/13.png}{}{13}
	\clearpage
	\fg{rys/12. Wordpress/14.png}{}{14}
	\clearpage
	\fg{rys/12. Wordpress/15.png}{}{15}
	\clearpage
	\fg{rys/12. Wordpress/16.png}{}{16}
	\clearpage
	\fg{rys/12. Wordpress/17.png}{}{17}
	\clearpage
	\fg{rys/12. Wordpress/18.png}{}{18}
	\clearpage
	\fg{rys/12. Wordpress/19.png}{}{19}
	\clearpage
	\fg{rys/12. Wordpress/20.png}{}{20}
	\clearpage
	\fg{rys/12. Wordpress/21.png}{}{21}
	\clearpage
	\fg{rys/12. Wordpress/22.png}{}{22}
	\clearpage
	\fg{rys/12. Wordpress/23.png}{}{23}
	\clearpage
	\fg{rys/12. Wordpress/24.png}{}{24}
	\clearpage
	\fg{rys/12. Wordpress/25.png}{}{25}
	\clearpage
	\fg{rys/12. Wordpress/26.png}{}{26}
	\clearpage
	\fg{rys/12. Wordpress/27.png}{}{27}
	\clearpage
	\fg{rys/12. Wordpress/28.png}{}{28}
	\clearpage
	\fg{rys/12. Wordpress/29.png}{}{29}
	\clearpage
	\fg{rys/12. Wordpress/30.png}{}{30}
	\clearpage
	\fg{rys/12. Wordpress/31.png}{}{31}
	\clearpage
	\fg{rys/12. Wordpress/32.png}{}{32}
	\clearpage
	\fg{rys/12. Wordpress/33.png}{}{33}
	\clearpage
	\fg{rys/12. Wordpress/34.png}{}{34}
	\clearpage
	\fg{rys/12. Wordpress/35.png}{}{35}
	\clearpage
	\fg{rys/12. Wordpress/36.png}{}{36}
	\clearpage
	\fg{rys/12. Wordpress/37.png}{}{37}
	\clearpage
	\fg{rys/12. Wordpress/38.png}{}{38}
	\clearpage
	\fg{rys/12. Wordpress/39.png}{}{39}
	\clearpage
	\fg{rys/12. Wordpress/40.png}{}{40}
	\clearpage
	\fg{rys/12. Wordpress/41.png}{}{41}
	\clearpage
	\fg{rys/12. Wordpress/42.png}{}{42}
	\clearpage
	\fg{rys/12. Wordpress/43.png}{}{43}
	\clearpage
	\fg{rys/12. Wordpress/44.png}{}{44}
	\clearpage
	\fg{rys/12. Wordpress/45.png}{}{45}
	\clearpage
	\fg{rys/12. Wordpress/46.png}{}{46}
	\clearpage
	\fg{rys/12. Wordpress/47.png}{}{47}
	\clearpage
	\fg{rys/12. Wordpress/48.png}{}{48}
	\clearpage
	\fg{rys/12. Wordpress/49.png}{}{49}
	\clearpage
	\fg{rys/12. Wordpress/50.png}{}{50}
	\clearpage
	\fg{rys/12. Wordpress/51.png}{}{51}
	\clearpage
	\fg{rys/12. Wordpress/52.png}{}{52}
	\clearpage
	\fg{rys/12. Wordpress/53.png}{}{53}
	\clearpage
	\fg{rys/12. Wordpress/54.png}{}{54}
	\clearpage
	\fg{rys/12. Wordpress/55.png}{}{55}
	\clearpage
	\fg{rys/12. Wordpress/56.png}{}{56}
	\clearpage
	\fg{rys/12. Wordpress/57.png}{}{57}
	\clearpage
	\fg{rys/12. Wordpress/58.png}{}{58}
	\clearpage
	\fg{rys/12. Wordpress/59.png}{}{59}
	\clearpage
	\fg{rys/12. Wordpress/60.png}{}{60}
	\clearpage
	\fg{rys/12. Wordpress/61.png}{}{61}
	\clearpage
	\fg{rys/12. Wordpress/62.png}{}{62}
	\clearpage
	\fg{rys/12. Wordpress/63.png}{}{63}
	\clearpage
	\fg{rys/12. Wordpress/64.png}{}{64}
	\clearpage
	\fg{rys/12. Wordpress/65.png}{}{65}
	\clearpage
	\fg{rys/12. Wordpress/66.png}{}{66}
	\clearpage
	\fg{rys/12. Wordpress/67.png}{}{67}
	\clearpage
	\fg{rys/12. Wordpress/68.png}{}{68}
	\clearpage
	\fg{rys/12. Wordpress/69.png}{}{69}
	\clearpage
	\fg{rys/12. Wordpress/70.png}{}{70}
	\clearpage
	\fg{rys/12. Wordpress/71.png}{}{71}
	\clearpage
	\fg{rys/12. Wordpress/72.png}{}{72}
	\clearpage

   	\newpage
\section{Testy działania wdrożonych usług}	%5
% (W podpunktach zamieścić zrzuty ekranów pokazujące działanie wdrożonych usług)


   \newpage
\section{Wnioski}	%5
%Npisać wnioski końcowe z przeprowadzonego projektu,

\subsection{Podsumowanie projektu}
Projekt zakładający stworzenie kompleksowej infrastruktury sieciowej dla przedsiębiorstwa został pomyślnie zrealizowany. Wdrożenie domeny Active Directory, usług katalogowych oraz systemów wsparcia pracy grupowej przyczyniło się do zwiększenia efektywności zarządzania zasobami IT oraz poprawy bezpieczeństwa danych.

\subsection{Korzyści z wdrożenia}
\begin{itemize}
    \item \textbf{Centralizacja zarządzania:} Dzięki wdrożeniu Active Directory możliwe jest centralne zarządzanie użytkownikami, komputerami i zasobami sieciowymi, co znacznie ułatwia administrację i redukuje koszty operacyjne.
    \item \textbf{Zwiększenie bezpieczeństwa:} Implementacja mechanizmów uwierzytelniania i autoryzacji, takich jak GPO i firewall, zapewnia wysoki poziom ochrony danych przed nieautoryzowanym dostępem i atakami.
    \item \textbf{Automatyzacja procesów:} Automatyczna instalacja systemów operacyjnych, konfiguracja stacji roboczych oraz zarządzanie zasobami sieciowymi przyspiesza wdrażanie nowych urządzeń i minimalizuje ryzyko błędów konfiguracyjnych.
    \item \textbf{Wysoka dostępność:} Klaster wysokiej dostępności oraz macierz RAID-1 zapewniają ciągłość działania usług i minimalizują ryzyko utraty danych w przypadku awarii sprzętu.
    \item \textbf{Elastyczność i skalowalność:} Wykorzystanie technologii takich jak iSCSI i WDS umożliwia łatwe skalowanie infrastruktury IT w miarę wzrostu potrzeb przedsiębiorstwa.
\end{itemize}

\subsection{Wyzwania i trudności}
Podczas realizacji projektu napotkano kilka wyzwań, które wymagały szczególnej uwagi:
\begin{itemize}
    \item \textbf{Integracja systemów:} Integracja różnych technologii, takich jak IIS, WordPress i Active Directory, wymagała precyzyjnej konfiguracji i testowania, aby zapewnić ich poprawne działanie.
    \item \textbf{Zarządzanie zmianami:} Wprowadzenie nowych rozwiązań technologicznych wiązało się z koniecznością przeszkolenia personelu oraz dostosowania istniejących procedur operacyjnych.
    \item \textbf{Bezpieczeństwo:} Zapewnienie odpowiedniego poziomu bezpieczeństwa danych i systemów wymagało implementacji zaawansowanych mechanizmów ochrony oraz regularnego monitorowania i aktualizacji.
\end{itemize}

\subsection{Rekomendacje na przyszłość}
Na podstawie doświadczeń z realizacji projektu, przedstawiamy kilka rekomendacji na przyszłość:
\begin{itemize}
    \item \textbf{Regularne szkolenia:} Przeprowadzanie regularnych szkoleń dla personelu IT oraz użytkowników końcowych w zakresie nowych technologii i procedur bezpieczeństwa.
    \item \textbf{Monitorowanie i audyt:} Wdrożenie systemów monitorowania i audytu, które pozwolą na bieżące śledzenie stanu infrastruktury IT oraz wykrywanie potencjalnych zagrożeń i nieprawidłowości.
    \item \textbf{Planowanie rozwoju:} Opracowanie długoterminowego planu rozwoju infrastruktury IT, uwzględniającego przyszłe potrzeby przedsiębiorstwa oraz możliwości technologiczne.
    \item \textbf{Zarządzanie ryzykiem:} Implementacja procedur zarządzania ryzykiem, które pozwolą na szybkie reagowanie na incydenty oraz minimalizowanie ich wpływu na działalność przedsiębiorstwa.
\end{itemize}

\subsection{Wnioski końcowe}
Realizacja projektu przyniosła wymierne korzyści dla przedsiębiorstwa, zarówno w zakresie efektywności zarządzania zasobami IT, jak i poprawy bezpieczeństwa danych. Wdrożone rozwiązania technologiczne zapewniają elastyczność i skalowalność infrastruktury, co pozwala na dalszy rozwój i adaptację do zmieniających się potrzeb biznesowych. Dalsze doskonalenie procesów oraz inwestycje w nowe technologie będą kluczowe dla utrzymania konkurencyjności i zapewnienia ciągłości działania przedsiębiorstwa.



   
       
%%%%%%%%%%%%%%%%%%% koniec treść główna dokumentu %%%%%%%%%%%%%%%%%%%%%
	\newpage
    % \addcontentsline{toc}{section}{Literatura}
    % Modified by: Maciej Wójs  
    \printbibliography[heading=bibnumbered, label=Literatura, title=Literatura]

    \newpage
    \hypersetup{linkcolor=black}
    \renewcommand{\cftparskip}{3pt}
    \clearpage
    \renewcommand{\cftloftitlefont}{\Large\bfseries\sffamily}
    \listoffigures
    \addcontentsline{toc}{section}{Spis rysunków}
	\thispagestyle{fancy}
	
    \newpage
    \renewcommand{\cftlottitlefont}{\Large\bfseries\sffamily}
    \def\listtablename{Spis tabel}
    \addcontentsline{toc}{section}{Spis tabel}\listoftables 
	\thispagestyle{fancy}
	
	\newpage
	\renewcommand{\cftlottitlefont}{\Large\bfseries\sffamily}
	\renewcommand\lstlistlistingname{Spis listingów}
	\addcontentsline{toc}{section}{Spis listingów}\lstlistoflistings 
	\thispagestyle{fancy}
    \label{LastPage}
	


    %lista rzeczy do zrobienia: wypisuje na koñcu dokumentu, patrz: pakiet todo.sty
    \todos
    %koniec listy rzeczy do zrobienia
\end{document}
