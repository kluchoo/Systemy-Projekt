%╔════════════════════════════╗
%║	  Szablon dostosował	  ║
%║	mgr inż. Dawid Kotlarski  ║
%║		  06.10.2024		  ║
%╚════════════════════════════╝
\documentclass[12pt,twoside,a4paper,openany]{article}

    % ------------------------------------------------------------------------
% PAKIETY
% ------------------------------------------------------------------------

%różne pakiety matematyczne, warto przejrzeć dokumentację, muszą być powyżej ustawień językowych.
\usepackage{mathrsfs}   %Różne symbole matematyczne opisane w katalogu ~\doc\latex\comprehensive. Zamienia \mathcal{L} ze zwykłego L na L-transformatę.
\usepackage{eucal}      %Różne symbole matematyczne.
\usepackage{amssymb}    %Różne symbole matematyczne.
\usepackage{amsmath}    %Dodatkowe funkcje matematyczne, np. polecenie \dfac{}{} skladajace ulamek w trybie wystawionym (porównaj $\dfrac{1}{2}$, a $\frac{1}{2}$).
\usepackage{lastpage}
%język polski i klawiatura
\usepackage[polish]{babel}
%\usepackage{qtimes} % czcionka Times new Roman
\usepackage[OT4]{polski}
%\usepackage[cp1250]{inputenc}                       %Strona kodowa polskich znaków.

%obsługa pdf'a
\usepackage[pdftex,usenames,dvipsnames]{color}      %Obsługa kolorów. Opcje usenames i dvipsnames wprowadzają dodatkowe nazwy kolorow.
\usepackage[pdftex,pagebackref=false,draft=false,pdfpagelabels=false,colorlinks=true,urlcolor=blue,linkcolor=black,citecolor=green,pdfstartview=FitH,pdfstartpage=1,pdfpagemode=UseOutlines,bookmarks=true,bookmarksopen=true,bookmarksopenlevel=2,bookmarksnumbered=true,pdfauthor={Dawid Kotlarski},pdftitle={Dokumentacja Projektowa},pdfsubject={},pdfkeywords={transient recovery voltage trv},unicode=true]{hyperref}   %Opcja pagebackref=true dotyczy bibliografii: pokazuje w spisie literatury numery stron, na których odwołano się do danej pozycji.

%bibliografia
%\usepackage[numbers,sort&compress]{natbib}  %Porządkuje zawartość odnośników do literatury, np. [2-4,6]. Musi być pod pdf'em, a styl bibliogfafii musi mieć nazwę z dodatkiem 'nat', np. \bibliographystyle{unsrtnat} (w kolejności cytowania).
\usepackage[
backend=biber,
style=numeric,
sorting=none
]{biblatex}
\addbibresource{bibliografia.bib}
\usepackage{hypernat}                       %Potrzebna pakietowi natbib do wspolpracy z pakietem hyperref (wazna kolejnosc: 1. hyperref, 2. natbib, 3. hypernat).

%grafika i geometria strony
\usepackage{extsizes}           %Dostepne inne rozmiary czcionek, np. 14 w poleceniu: \documentclass[14pt]{article}.
\usepackage[final]{graphicx}
\usepackage[a4paper,left=3.5cm,right=2.5cm,top=2.5cm,bottom=2.5cm]{geometry}

%strona tytułowa
\usepackage{strona_tytulowa}

%inne
\usepackage[hide]{todo}                     %Wprowadza polecenie \todo{treść}. Opcje pakietu: hide/show. Polecenie \todos ma byc na koncu dokumentu, wszystkie \todo{} po \todos sa ignorowane.
\usepackage[basic,physics]{circ}            %Wprowadza środowisko circuit do rysowania obwodów elektrycznych. Musi byc poniżej pakietow językowych.
\usepackage[sf,bf,outermarks]{titlesec}     %Troszczy się o wygląd tytułów rozdziałów (section, subsection, ...). sf oznacza czcionkę sans serif (typu arial), bf -- bold. U mnie: oddzielna linia dla naglowku paragraph. Patrz tez: tocloft -- lepiej robi format spisu tresci.
\usepackage{tocloft}                        %Troszczy się o format spisu trsci.
\usepackage{expdlist}    %Zmienia definicję środowiska description, daje większe możliwości wpływu na wygląd listy.
\usepackage{flafter}     %Wprowadza parametr [tb] do polecenia \suppressfloats[t] (polecenie to powoduje nie umieszczanie rysunkow, tabel itp. na stronach, na ktorych jest to polecenie (np. moze byc to stroma z tytulem rozdzialu, ktory chcemy zeby byl u samej gory, a nie np. pod rysunkiem)).
\usepackage{array}       %Ładniej drukuje tabelki (np. daje wiecej miejsca w komorkach -- nie są tak ścieśnione, jak bez tego pakietu).
\usepackage{listings}    %Listingi programow.
\usepackage[format=hang,labelsep=period,labelfont={bf,small},textfont=small]{caption}   %Formatuje podpisy pod rysunkami i tabelami. Parametr 'hang' powoduje wcięcie kolejnych linii podpisu na szerokosc nazwy podpisu, np. 'Rysunek 1.'.
\usepackage{appendix}    %Troszczy się o załączniki.
\usepackage{floatflt}    %Troszczy się o oblewanie rysunkow tekstem.
\usepackage{here}        %Wprowadza dodtkowy parametr umiejscowienia rysunków, tabel, itp.: H (duże). Umiejscawia obiekty ruchome dokladnie tam gdzie są w kodzie źródłowym dokumentu.
\usepackage{makeidx}     %Troszczy się o indeks (skorowidz).

%nieużywane, ale potencjalnie przydatne
\usepackage{sectsty}           %Formatuje nagłówki, np. żeby były kolorowe -- polecenie: \allsectionsfont{\color{Blue}}.
%\usepackage{version}           %Wersje dokumentu.

%============
\usepackage{longtable}			%tabelka
%============

%============
% Ustawienia listingów do kodu
%============

\usepackage{listings}
\usepackage{xcolor}

\definecolor{codegreen}{rgb}{0,0.6,0}
\definecolor{codegray}{rgb}{0.5,0.5,0.5}
\definecolor{codepurple}{rgb}{0.58,0,0.82}
\definecolor{backcolour}{rgb}{0.95,0.95,0.92}

% Definicja stylu "mystyle"
\lstdefinestyle{mystyle}{
	backgroundcolor=\color{backcolour},   
	commentstyle=\color{codegreen},
	keywordstyle=\color{blue},	%magenta
	numberstyle=\tiny\color{codegray},
	stringstyle=\color{codepurple},
	basicstyle=\ttfamily\footnotesize,
	breakatwhitespace=false,         
	breaklines=true,                 
	captionpos=b,                    
	keepspaces=true,                 
	numbers=left,                    
	numbersep=5pt,                  
	showspaces=false,                
	showstringspaces=false,
	showtabs=false,                  
	tabsize=2,
	literate=
  {á}{{\'a}}1 {é}{{\'e}}1 {í}{{\'i}}1 {ó}{{\'o}}1 {ú}{{\'u}}1
  {Á}{{\'A}}1 {É}{{\'E}}1 {Í}{{\'I}}1 {Ó}{{\'O}}1 {Ú}{{\'U}}1
  {à}{{\`a}}1 {è}{{\`e}}1 {ì}{{\`i}}1 {ò}{{\`o}}1 {ù}{{\`u}}1
  {À}{{\`A}}1 {È}{{\`E}}1 {Ì}{{\`I}}1 {Ò}{{\`O}}1 {Ù}{{\`U}}1
  {ä}{{\"a}}1 {ë}{{\"e}}1 {ï}{{\"i}}1 {ö}{{\"o}}1 {ü}{{\"u}}1
  {Ä}{{\"A}}1 {Ë}{{\"E}}1 {Ï}{{\"I}}1 {Ö}{{\"O}}1 {Ü}{{\"U}}1
  {â}{{\^a}}1 {ê}{{\^e}}1 {î}{{\^i}}1 {ô}{{\^o}}1 {û}{{\^u}}1
  {Â}{{\^A}}1 {Ê}{{\^E}}1 {Î}{{\^I}}1 {Ô}{{\^O}}1 {Û}{{\^U}}1
  {ã}{{\~a}}1 {ẽ}{{\~e}}1 {ĩ}{{\~i}}1 {õ}{{\~o}}1 {ũ}{{\~u}}1
  {Ã}{{\~A}}1 {Ẽ}{{\~E}}1 {Ĩ}{{\~I}}1 {Õ}{{\~O}}1 {Ũ}{{\~U}}1
  {œ}{{\oe}}1 {Œ}{{\OE}}1 {æ}{{\ae}}1 {Æ}{{\AE}}1 {ß}{{\ss}}1
  {ű}{{\H{u}}}1 {Ű}{{\H{U}}}1 {ő}{{\H{o}}}1 {Ő}{{\H{O}}}1
  {ç}{{\c c}}1 {Ç}{{\c C}}1 {ø}{{\o}}1 {Ø}{{\O}}1 {å}{{\r a}}1 {Å}{{\r A}}1
  {€}{{\euro}}1 {£}{{\pounds}}1 {«}{{\guillemotleft}}1
  {»}{{\guillemotright}}1 {ñ}{{\~n}}1 {Ñ}{{\~N}}1 {¿}{{?`}}1 {¡}{{!`}}1 
  {ą}{{\k{a}}}1 {ć}{{\'{c}}}1 {ę}{{\k{e}}}1 {ł}{{\l}}1 {ń}{{\'n}}1 
  {ó}{{\'o}}1 {ś}{{\'s}}1 {ź}{{\'z}}1 {ż}{{\.{z}}}1 
  {Ą}{{\k{A}}}1 {Ć}{{\'{C}}}1 {Ę}{{\k{E}}}1 {Ł}{{\L}}1 {Ń}{{\'N}}1
  {Ó}{{\'O}}1 {Ś}{{\'S}}1 {Ź}{{\'Z}}1 {Ż}{{\.{Z}}}1
}

\lstset{style=mystyle} % Deklaracja aktywnego stylu
%===========

%PAGINA GÓRNA I DOLNA
\usepackage{fancyhdr}          %Dodaje naglowki jakie się chce.
\pagestyle{fancy}
\fancyhf{}
% numery stron w paginie dolnej na srodku
\fancyfoot[C]{\footnotesize DOKUMENTACJA PROJEKTU – SYSTEMY OPERACYJNE  \\ 
\normalsize\sffamily  \thepage\ z~\pageref{LastPage}}


%\fancyhead[L]{\small\sffamily \nouppercase{\leftmark}}
\fancyhead[C]{\footnotesize \textit{AKADEMIA NAUK STOSOWANYCH W NOWYM SĄCZU}\\}

\renewcommand{\headrulewidth}{0.4pt}
\renewcommand{\footrulewidth}{0.4pt}

    % ------------------------------------------------------------------------
% USTAWIENIA
% ------------------------------------------------------------------------

% ------------------------------------------------------------------------
%   Kropki po numerach sekcji, podsekcji, itd.
%   Np. 1.2. Tytuł podrozdziału
% ------------------------------------------------------------------------
\makeatletter
    \def\numberline#1{\hb@xt@\@tempdima{#1.\hfil}}                      %kropki w spisie treści
    \renewcommand*\@seccntformat[1]{\csname the#1\endcsname.\enspace}   %kropki w treści dokumentu
\makeatother

% ------------------------------------------------------------------------
%   Numeracja równań, rysunków i tabel
%   Np.: (1.2), gdzie:
%   1 - numer sekcji, 2 - numer równania, rysunku, tabeli
%   Uwaga ogólna: o otoczeniu figure ma być najpierw \caption{}, potem \label{}, inaczej odnośnik nie działa!
% ------------------------------------------------------------------------
\makeatletter
    \@addtoreset{equation}{section} %resetuje licznik po rozpoczęciu nowej sekcji
    \renewcommand{\theequation}{{\thesection}.\@arabic\c@equation} %dodaje kropki

    \@addtoreset{figure}{section}
    \renewcommand{\thefigure}{{\thesection}.\@arabic\c@figure}

    \@addtoreset{table}{section}
    \renewcommand{\thetable}{{\thesection}.\@arabic\c@table}
\makeatother

% ------------------------------------------------------------------------
% Tablica
% ------------------------------------------------------------------------
\newenvironment{tabela}[3]
{
    \begin{table}[!htb]
    \centering
    \caption[#1]{#2}
    \vskip 9pt
    #3
}{
    \end{table}
}

% ------------------------------------------------------------------------
% Dostosowanie wyglądu pozycji listy \todos, np. zamiast 'p.' jest 'str.'
% ------------------------------------------------------------------------
\renewcommand{\todoitem}[2]{%
    \item \label{todo:\thetodo}%
    \ifx#1\todomark%
        \else\textbf{#1 }%
    \fi%
    (str.~\pageref{todopage:\thetodo})\ #2}
\renewcommand{\todoname}{Do zrobienia...}
\renewcommand{\todomark}{~uzupełnić}

% ------------------------------------------------------------------------
% Definicje
% ------------------------------------------------------------------------
\def\nonumsection#1{%
    \section*{#1}%
    \addcontentsline{toc}{section}{#1}%
    }
\def\nonumsubsection#1{%
    \subsection*{#1}%
    \addcontentsline{toc}{subsection}{#1}%
    }
\reversemarginpar %umieszcza notki po lewej stronie, czyli tam gdzie jest więcej miejsca
\def\notka#1{%
    \marginpar{\footnotesize{#1}}%
    }
\def\mathcal#1{%
    \mathscr{#1}%
    }
\newcommand{\atp}{ATP/EMTP} % Inaczej: \def\atp{ATP/EMTP}

% ------------------------------------------------------------------------
% Inne
% ------------------------------------------------------------------------
\frenchspacing                      
\hyphenation{ATP/-EMTP}             %dzielenie wyrazu w danym miejscu
\setlength{\parskip}{3pt}           %odstęp pomiędzy akapitami
\linespread{1.3}                    %odstęp pomiędzy liniami (interlinia)
\setcounter{tocdepth}{4}            %uwzględnianie w spisie treści czterech poziomów sekcji
\setcounter{secnumdepth}{4}         %numerowanie do czwartego poziomu sekcji 
\titleformat{\paragraph}[hang]      %wygląd nagłówków
{\normalfont\sffamily\bfseries}{\theparagraph}{1em}{}

%komenda do łatwiejszego wstawiania zdjęć
\newcommand*{\fg}[4][\textwidth]{
    \begin{figure}[!htb]
        \begin{center}
            \includegraphics[width=#1]{#2}
            \caption{#3}
            \label{rys:#4}
        \end{center}
    \end{figure}
}

\newcommand*{\Oznacz}[2]{
\ref{#1:#2} (s. \pageref{#1:#2})
}

\newcommand*{\OznaczZdjecie}[2][Rysunek]{
#1 \Oznacz{rys}{#2}
}
    
\newcommand*{\OznaczKod}[1]{
\Oznacz{lst}{#1}
}

\newcommand*{\ListingFile}[2]{
    \lstinputlisting[caption=#1, label={lst:#2}, language=C++]{kod/#2.txt}
}


    %polecenia zdefiniowane w pakiecie strona_tytulowa.sty
    \title{Zaprojektować i wdrożyć system informatyczny na
    potrzeby przedsiębiorstwa zgodnie z założeniami}		%...Wpisać nazwę projektu...
    \author{Imie Nazwisko}
    \authorI{}
    \authorII{}		%jeśli są dwie osoby w projekcie to zostawiamy:    \authorII{}
		
	\uczelnia{AKADEMIA NAUK STOSOWANYCH \\W NOWYM SĄCZU}
    \instytut{Wydział Nauk Inżynieryjnych}
    \kierunek{Katedra Informatyki}
    \praca{DOKUMENTACJA PROJEKTOWA}
    \przedmiot{SYSTEMY OPERACYJNE}
    \prowadzacy{mgr inż. Jan Kozieński}
    \rok{2025}


%definicja składni mikrotik
\usepackage{fancyvrb}
\DefineVerbatimEnvironment{MT}{Verbatim}%
{commandchars=\+\[\],fontsize=\small,formatcom=\color{red},frame=lines,baselinestretch=1,} 
\let\mt\verb 
%zakonczenie definicji składni mikrotik

\usepackage{fancyhdr}    %biblioteka do nagłówka i stopki

			
\begin{document}
   
    \renewcommand{\figurename}{Rys.}    %musi byc pod \begin{document}, bo w~tym miejscu pakiet 'babel' narzuca swoje ustawienia
    \renewcommand{\tablename}{Tab.}     %j.w.
    \thispagestyle{empty}               %na tej stronie: brak numeru
    \stronatytulowa                     %strona tytułowa tworzona przez pakiet strona_tytulowa.tex
 
 \pagestyle{fancy}

    \newpage

    %formatowanie spisu treści i~nagłówków
    \renewcommand{\cftbeforesecskip}{8pt}
    \renewcommand{\cftsecafterpnum}{\vskip 8pt}
    \renewcommand{\cftparskip}{3pt}
    \renewcommand{\cfttoctitlefont}{\Large\bfseries\sffamily}
    \renewcommand{\cftsecfont}{\bfseries\sffamily}
    \renewcommand{\cftsubsecfont}{\sffamily}
    \renewcommand{\cftsubsubsecfont}{\sffamily}
    \renewcommand{\cftparafont}{\sffamily}
    %koniec formatowania spisu treści i nagłówków
     
    \tableofcontents    %spis treści
    \thispagestyle{fancy}
    \newpage

    
    \newpage

    
%%%%%%%%%%%%%%%%%%% treść główna dokumentu %%%%%%%%%%%%%%%%%%%%%%%%%

   %! SKRÓTY KLAWISZOWE
% LINK do skrótów klawiszowych: https://github.com/James-Yu/latex-workshop/wiki/Snippets
% ctrl+alt+j - przeniesienie z kodu do pdf
% ctrl + click - przeniesienie z pdf do kodu (dokument.pdf)
% zaznaczony fragment kodu -> ctrl+l -> ctrl+w
% gdy kuror na sekcji itp. -> cltr + alt + ] - obniżenie sekcji
% gdy kuror na sekcji itp. -> cltr + alt + [ - podniesienie sekcji
% kopia lini kodu -> ctrl + shift + strzałka w dół
\newpage
\section{Założenia projektowe – wymagania}		%1

\subsection{Wprowadzenie}
Projekt zakłada stworzenie kompleksowej infrastruktury sieciowej dla przedsiębiorstwa, obejmującej wdrożenie domeny Active Directory, usług katalogowych oraz systemów wsparcia pracy grupowej.

\subsection{Struktura organizacyjna}
Przedsiębiorstwo składa się z następujących wydziałów:
\begin{itemize}
    \item Wydział Informatyczny (administracja systemem)
    \item Kadry
    \item Płace
    \item Marketing
    \item Wydział Gospodarczy
\end{itemize}

\subsection{Wymagania infrastrukturalne}
\begin{enumerate}
    \item \textbf{Domena Active Directory:}
    \begin{itemize}
        \item Nazwa domeny: firma.ad
        \item Schemat nazewnictwa kont: imie.nazwisko
        \item Automatyzacja tworzenia kont poprzez skrypty
    \end{itemize}

    \item \textbf{Struktura grupowa:}
    \begin{itemize}
        \item Grupy globalne dla każdego wydziału
        \item Specjalne uprawnienia dla wydziału informatycznego
    \end{itemize}

    \item \textbf{Zasoby sieciowe:}
    \begin{itemize}
        \item Zasób wspólny dostępny dla wszystkich pracowników
        \item Zasoby wydziałowe dla poszczególnych działów
        \item Klaster wysokiej dostępności
        \item Macierz RAID-1 (30GB) udostępniana przez iSCSI
    \end{itemize}

    \item \textbf{Infrastruktura serwerowa:}
    \begin{itemize}
        \item Serwer SMPXX.firma.ad (iSCSI Target)
        \item Serwer SPRXX-firma.ad (serwer wydruków)
        \item Klaster DHCP
        \item Serwer WWW (IIS + WordPress)
    \end{itemize}

    \item \textbf{Stacje robocze:}
    \begin{itemize}
        \item Automatyczna instalacja z obrazu
        \item Automatyczna konfiguracja przez GPO
        \item Mapowanie dysków sieciowych
    \end{itemize}
\end{enumerate}

\subsection{Szczegółowe wymagania techniczne}
\begin{enumerate}
    \item \textbf{System drukowania:}
    \begin{itemize}
        \item Centralny serwer wydruków
        \item Dostęp do drukarek przez \\ SPRXX-firma.ad/nazwa-drukarki
    \end{itemize}

    \item \textbf{Usługi sieciowe:}
    \begin{itemize}
        \item DHCP w konfiguracji klastrowej
        \item Strona WWW pod adresem www.firma.ad
        \item WordPress zintegrowany z IIS
    \end{itemize}
\end{enumerate}

   \newpage
\section{Opis użytych technologii}		%2
%(W podpunktach dokonać krótkiej charakterystyki użytych technologii ) 



   	\newpage
\section{Schemat logiczny}		%3
% (Ma zawierać aktualne nazewnictwo i adresację IP.)
\fg{rys/Projekt.drawio.png}{Schemat logiczny projektu}{1}

   	\newpage
\section{Procedury instalacyjne poszczególnych usług}		%4
% Procedury instalacyjne poszczególnych usług.
% (W podpunktach zamieścić polecenia dotyczące instalacji wdrażanych usług) 

   	\newpage
\section{Testy działania wdrożonych usług}	%5
% (W podpunktach zamieścić zrzuty ekranów pokazujące działanie wdrożonych usług)

\subsection{Test 1: Logowanie do domeny Active Directory}
\fg{rys/2. Active Directory/24.png}{Logowanie do domeny Active Directory}{5_1}	
\clearpage
\fg{rys/2. Active Directory/25.png}{Lista użytkowników}{5_2}	
\clearpage
\subsection{Test 2: RAID1}
\fg{rys/3. raid1/14.png}{RAID1 - zarządzanie dyskami}{5_3}
\clearpage
\subsection{Test 3: iSCSI}
\fg{rys/4. iSCSI/52.png}{iSCSI - Widok działającego klastra}{5_4}
\clearpage
\subsection{Test 4: FS}
\fg{rys/5. FS/47.png}{FS - Działające foldery grupy}{5_5}
\clearpage
\subsection{Test 5: GPO}
\fg{rys/6. GPO/23.png}{GPO - Zmienna systemowa}{5_6}
\fg{rys/6. GPO/28.png}{GPO - Mapowanie folderu}{5_7}
\clearpage
\subsection{Test 6: DNS}
\fg{rys/7. DNS/14.png}{DNS - Rekordy}{5_8}
\clearpage
\subsection{Test 7: DHCP}
\fg{rys/8. DHCP/46.png}{DHCP - Adres ip przydzielony z DHCP}{5_9}
\clearpage
\subsection{Test 8: WDS}
\fg{rys/9. WDS/30.png}{WDS - Instalacja systemu}{5_10}
\clearpage
\subsection{Test 9: Serwer drukowania}
\fg{rys/10. Drukarka/18.png}{Serwer drukowania - Drukarki}{5_11}
\clearpage
\subsection{Test 10: Serwer IIS}
\fg{rys/11. IIS/19.png}{Serwer IIS - Strona główna}{5_12}
\fg{rys/11. IIS/20.png}{Serwer IIS - Strona główna na kliencie}{5_12}
\clearpage
\subsection{Test 11: Serwer WordPress}
\fg{rys/12. Wordpress/74.png}{Serwer WordPress - Strona główna}{5_13}

   \newpage
\section{Wnioski}	%5
%Npisać wnioski końcowe z przeprowadzonego projektu,

\subsection{Podsumowanie projektu}
Projekt zakładający stworzenie kompleksowej infrastruktury sieciowej dla przedsiębiorstwa został pomyślnie zrealizowany. Wdrożenie domeny Active Directory, usług katalogowych oraz systemów wsparcia pracy grupowej przyczyniło się do zwiększenia efektywności zarządzania zasobami IT oraz poprawy bezpieczeństwa danych.

\subsection{Korzyści z wdrożenia}
\begin{itemize}
    \item \textbf{Centralizacja zarządzania:} Dzięki wdrożeniu Active Directory możliwe jest centralne zarządzanie użytkownikami, komputerami i zasobami sieciowymi, co znacznie ułatwia administrację i redukuje koszty operacyjne.
    \item \textbf{Zwiększenie bezpieczeństwa:} Implementacja mechanizmów uwierzytelniania i autoryzacji, takich jak GPO i firewall, zapewnia wysoki poziom ochrony danych przed nieautoryzowanym dostępem i atakami.
    \item \textbf{Automatyzacja procesów:} Automatyczna instalacja systemów operacyjnych, konfiguracja stacji roboczych oraz zarządzanie zasobami sieciowymi przyspiesza wdrażanie nowych urządzeń i minimalizuje ryzyko błędów konfiguracyjnych.
    \item \textbf{Wysoka dostępność:} Klaster wysokiej dostępności oraz macierz RAID-1 zapewniają ciągłość działania usług i minimalizują ryzyko utraty danych w przypadku awarii sprzętu.
    \item \textbf{Elastyczność i skalowalność:} Wykorzystanie technologii takich jak iSCSI i WDS umożliwia łatwe skalowanie infrastruktury IT w miarę wzrostu potrzeb przedsiębiorstwa.
\end{itemize}

\subsection{Wyzwania i trudności}
Podczas realizacji projektu napotkano kilka wyzwań, które wymagały szczególnej uwagi:
\begin{itemize}
    \item \textbf{Integracja systemów:} Integracja różnych technologii, takich jak IIS, WordPress i Active Directory, wymagała precyzyjnej konfiguracji i testowania, aby zapewnić ich poprawne działanie.
    \item \textbf{Zarządzanie zmianami:} Wprowadzenie nowych rozwiązań technologicznych wiązało się z koniecznością przeszkolenia personelu oraz dostosowania istniejących procedur operacyjnych.
    \item \textbf{Bezpieczeństwo:} Zapewnienie odpowiedniego poziomu bezpieczeństwa danych i systemów wymagało implementacji zaawansowanych mechanizmów ochrony oraz regularnego monitorowania i aktualizacji.
\end{itemize}

\subsection{Rekomendacje na przyszłość}
Na podstawie doświadczeń z realizacji projektu, przedstawiamy kilka rekomendacji na przyszłość:
\begin{itemize}
    \item \textbf{Regularne szkolenia:} Przeprowadzanie regularnych szkoleń dla personelu IT oraz użytkowników końcowych w zakresie nowych technologii i procedur bezpieczeństwa.
    \item \textbf{Monitorowanie i audyt:} Wdrożenie systemów monitorowania i audytu, które pozwolą na bieżące śledzenie stanu infrastruktury IT oraz wykrywanie potencjalnych zagrożeń i nieprawidłowości.
    \item \textbf{Planowanie rozwoju:} Opracowanie długoterminowego planu rozwoju infrastruktury IT, uwzględniającego przyszłe potrzeby przedsiębiorstwa oraz możliwości technologiczne.
    \item \textbf{Zarządzanie ryzykiem:} Implementacja procedur zarządzania ryzykiem, które pozwolą na szybkie reagowanie na incydenty oraz minimalizowanie ich wpływu na działalność przedsiębiorstwa.
\end{itemize}

\subsection{Wnioski końcowe}
Realizacja projektu przyniosła wymierne korzyści dla przedsiębiorstwa, zarówno w zakresie efektywności zarządzania zasobami IT, jak i poprawy bezpieczeństwa danych. Wdrożone rozwiązania technologiczne zapewniają elastyczność i skalowalność infrastruktury, co pozwala na dalszy rozwój i adaptację do zmieniających się potrzeb biznesowych. Dalsze doskonalenie procesów oraz inwestycje w nowe technologie będą kluczowe dla utrzymania konkurencyjności i zapewnienia ciągłości działania przedsiębiorstwa.



   
       
%%%%%%%%%%%%%%%%%%% koniec treść główna dokumentu %%%%%%%%%%%%%%%%%%%%%
	\newpage
    % \addcontentsline{toc}{section}{Literatura}
    % Modified by: Maciej Wójs  
    \printbibliography[heading=bibnumbered, label=Literatura, title=Literatura]

    \newpage
    \hypersetup{linkcolor=black}
    \renewcommand{\cftparskip}{3pt}
    \clearpage
    \renewcommand{\cftloftitlefont}{\Large\bfseries\sffamily}
    \listoffigures
    \addcontentsline{toc}{section}{Spis rysunków}
	\thispagestyle{fancy}
	
    \newpage
    \renewcommand{\cftlottitlefont}{\Large\bfseries\sffamily}
    \def\listtablename{Spis tabel}
    \addcontentsline{toc}{section}{Spis tabel}\listoftables 
	\thispagestyle{fancy}
	
	\newpage
	\renewcommand{\cftlottitlefont}{\Large\bfseries\sffamily}
	\renewcommand\lstlistlistingname{Spis listingów}
	\addcontentsline{toc}{section}{Spis listingów}\lstlistoflistings 
	\thispagestyle{fancy}
    \label{LastPage}
	


    %lista rzeczy do zrobienia: wypisuje na koñcu dokumentu, patrz: pakiet todo.sty
    \todos
    %koniec listy rzeczy do zrobienia
\end{document}
