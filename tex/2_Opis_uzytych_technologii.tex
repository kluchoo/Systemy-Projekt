\newpage
\section{Opis użytych technologii}		%2
%(W podpunktach dokonać krótkiej charakterystyki użytych technologii )

\subsection{Active Directory}
Active Directory (AD) to usługa katalogowa opracowana przez firmę Microsoft dla systemów Windows. Umożliwia zarządzanie użytkownikami, komputerami i innymi zasobami w sieci. AD zapewnia centralne uwierzytelnianie i autoryzację, co ułatwia zarządzanie dużymi sieciami.

\subsection{IIS (Internet Information Services)}
Internet Information Services (IIS) to serwer WWW stworzony przez Microsoft. Umożliwia hostowanie stron internetowych oraz aplikacji webowych. IIS obsługuje różne technologie, takie jak ASP.NET, PHP i WordPress, co czyni go wszechstronnym narzędziem do wdrażania aplikacji internetowych.

\subsection{WordPress}
WordPress to popularny system zarządzania treścią (CMS), który umożliwia tworzenie i zarządzanie stronami internetowymi. Jest łatwy w użyciu i oferuje szeroką gamę wtyczek oraz motywów, co pozwala na dostosowanie strony do indywidualnych potrzeb.

\subsection{DHCP (Dynamic Host Configuration Protocol)}
Dynamic Host Configuration Protocol (DHCP) to protokół sieciowy, który automatycznie przydziela adresy IP oraz inne parametry konfiguracyjne urządzeniom w sieci. DHCP upraszcza zarządzanie adresacją IP, eliminując konieczność ręcznego przypisywania adresów.

\subsection{RAID (Redundant Array of Independent Disks)}
RAID to technologia, która łączy wiele dysków twardych w jedną logiczną jednostkę w celu zwiększenia wydajności i/lub zapewnienia redundancji danych. RAID-1, znany również jako mirroring, polega na duplikowaniu danych na dwóch dyskach, co zapewnia ochronę przed utratą danych w przypadku awarii jednego z dysków.

\subsection{iSCSI (Internet Small Computer Systems Interface)}
iSCSI to protokół sieciowy, który umożliwia przesyłanie poleceń SCSI przez sieci IP. Dzięki iSCSI można tworzyć rozproszone systemy pamięci masowej, które są dostępne przez sieć, co pozwala na centralizację zarządzania danymi i zwiększenie elastyczności infrastruktury IT.

\subsection{GPO (Group Policy Objects)}
Group Policy Objects (GPO) to funkcja systemów Windows, która umożliwia centralne zarządzanie konfiguracją i ustawieniami komputerów oraz użytkowników w domenie Active Directory. GPO pozwala na automatyzację wielu zadań administracyjnych, takich jak instalacja oprogramowania, konfiguracja systemu i ustawienia zabezpieczeń.

\subsection{DNS (Domain Name System)}
Domain Name System (DNS) to system, który przyporządkowuje adresy IP do nazw domenowych. DNS ułatwia korzystanie z zasobów sieciowych, ponieważ pozwala na odwoływanie się do serwerów i usług za pomocą nazw zamiast adresów IP. Dzięki DNS możliwe jest również tworzenie hierarchicznej struktury domen, co ułatwia zarządzanie dużymi sieciami.

\subsection{FS (File Server)}
File Server (FS) to serwer, który umożliwia przechowywanie i udostępnianie plików w sieci. FS zapewnia centralizację danych, co ułatwia zarządzanie zasobami oraz zapewnia bezpieczeństwo i dostępność danych. FS może być wykorzystywany do przechowywania plików użytkowników, kopii zapasowych oraz innych danych firmowych.

\subsection{WDS (Windows Deployment Services)}
Windows Deployment Services (WDS) to narzędzie, które umożliwia zdalną instalację systemu Windows na stacjach roboczych. WDS pozwala na automatyzację procesu instalacji, co przyspiesza wdrażanie nowych komputerów oraz ułatwia zarządzanie systemami w sieci. WDS współpracuje z technologią PXE, która umożliwia uruchamianie systemu z sieci.

\subsection{iSCSI Target}
iSCSI Target to serwer, który udostępnia pamięć masową za pomocą protokołu iSCSI. iSCSI Target pozwala na tworzenie wirtualnych dysków, które są dostępne przez sieć i mogą być wykorzystywane przez inne serwery i stacje robocze. Dzięki iSCSI Target możliwe jest tworzenie rozproszonych systemów pamięci masowej, które są łatwo skalowalne i elastyczne. 
