\newpage
\section{Wnioski}	%5
%Npisać wnioski końcowe z przeprowadzonego projektu,

\subsection{Podsumowanie projektu}
Projekt zakładający stworzenie kompleksowej infrastruktury sieciowej dla przedsiębiorstwa został pomyślnie zrealizowany. Wdrożenie domeny Active Directory, usług katalogowych oraz systemów wsparcia pracy grupowej przyczyniło się do zwiększenia efektywności zarządzania zasobami IT oraz poprawy bezpieczeństwa danych.

\subsection{Korzyści z wdrożenia}
\begin{itemize}
    \item \textbf{Centralizacja zarządzania:} Dzięki wdrożeniu Active Directory możliwe jest centralne zarządzanie użytkownikami, komputerami i zasobami sieciowymi, co znacznie ułatwia administrację i redukuje koszty operacyjne.
    \item \textbf{Zwiększenie bezpieczeństwa:} Implementacja mechanizmów uwierzytelniania i autoryzacji, takich jak GPO i firewall, zapewnia wysoki poziom ochrony danych przed nieautoryzowanym dostępem i atakami.
    \item \textbf{Automatyzacja procesów:} Automatyczna instalacja systemów operacyjnych, konfiguracja stacji roboczych oraz zarządzanie zasobami sieciowymi przyspiesza wdrażanie nowych urządzeń i minimalizuje ryzyko błędów konfiguracyjnych.
    \item \textbf{Wysoka dostępność:} Klaster wysokiej dostępności oraz macierz RAID-1 zapewniają ciągłość działania usług i minimalizują ryzyko utraty danych w przypadku awarii sprzętu.
    \item \textbf{Elastyczność i skalowalność:} Wykorzystanie technologii takich jak iSCSI i WDS umożliwia łatwe skalowanie infrastruktury IT w miarę wzrostu potrzeb przedsiębiorstwa.
\end{itemize}

\subsection{Wyzwania i trudności}
Podczas realizacji projektu napotkano kilka wyzwań, które wymagały szczególnej uwagi:
\begin{itemize}
    \item \textbf{Integracja systemów:} Integracja różnych technologii, takich jak IIS, WordPress i Active Directory, wymagała precyzyjnej konfiguracji i testowania, aby zapewnić ich poprawne działanie.
    \item \textbf{Zarządzanie zmianami:} Wprowadzenie nowych rozwiązań technologicznych wiązało się z koniecznością przeszkolenia personelu oraz dostosowania istniejących procedur operacyjnych.
    \item \textbf{Bezpieczeństwo:} Zapewnienie odpowiedniego poziomu bezpieczeństwa danych i systemów wymagało implementacji zaawansowanych mechanizmów ochrony oraz regularnego monitorowania i aktualizacji.
\end{itemize}

\subsection{Rekomendacje na przyszłość}
Na podstawie doświadczeń z realizacji projektu, przedstawiamy kilka rekomendacji na przyszłość:
\begin{itemize}
    \item \textbf{Regularne szkolenia:} Przeprowadzanie regularnych szkoleń dla personelu IT oraz użytkowników końcowych w zakresie nowych technologii i procedur bezpieczeństwa.
    \item \textbf{Monitorowanie i audyt:} Wdrożenie systemów monitorowania i audytu, które pozwolą na bieżące śledzenie stanu infrastruktury IT oraz wykrywanie potencjalnych zagrożeń i nieprawidłowości.
    \item \textbf{Planowanie rozwoju:} Opracowanie długoterminowego planu rozwoju infrastruktury IT, uwzględniającego przyszłe potrzeby przedsiębiorstwa oraz możliwości technologiczne.
    \item \textbf{Zarządzanie ryzykiem:} Implementacja procedur zarządzania ryzykiem, które pozwolą na szybkie reagowanie na incydenty oraz minimalizowanie ich wpływu na działalność przedsiębiorstwa.
\end{itemize}

\subsection{Wnioski końcowe}
Realizacja projektu przyniosła wymierne korzyści dla przedsiębiorstwa, zarówno w zakresie efektywności zarządzania zasobami IT, jak i poprawy bezpieczeństwa danych. Wdrożone rozwiązania technologiczne zapewniają elastyczność i skalowalność infrastruktury, co pozwala na dalszy rozwój i adaptację do zmieniających się potrzeb biznesowych. Dalsze doskonalenie procesów oraz inwestycje w nowe technologie będą kluczowe dla utrzymania konkurencyjności i zapewnienia ciągłości działania przedsiębiorstwa.


